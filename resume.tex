%%%%%%%%%%%%%%%%%%%%%%%%%%%%%%%%%%%%%%%%%
% Developer CV
% LaTeX Template
% Version 1.0 (28/1/19)
%
% This template originates from:
% http://www.LaTeXTemplates.com
%
% Authors:
% Jan Vorisek (jan@vorisek.me)
% Based on a template by Jan Küster (info@jankuester.com)
% Modified for LaTeX Templates by Vel (vel@LaTeXTemplates.com)
%
% License:
% The MIT License (see included LICENSE file)
%
%%%%%%%%%%%%%%%%%%%%%%%%%%%%%%%%%%%%%%%%%

%----------------------------------------------------------------------------------------
%	PACKAGES AND OTHER DOCUMENT CONFIGURATIONS
%----------------------------------------------------------------------------------------

\documentclass[9pt]{developercv} % Default font size, values from 8-12pt are recommended

%----------------------------------------------------------------------------------------

\begin{document}

%----------------------------------------------------------------------------------------
%	TITLE AND CONTACT INFORMATION
%----------------------------------------------------------------------------------------

\begin{minipage}[t]{0.45\textwidth} % 45% of the page width for name
	\vspace{-\baselineskip} % Required for vertically aligning minipages
	
	% If your name is very short, use just one of the lines below
	% If your name is very long, reduce the font size or make the minipage wider and reduce the others proportionately
	\colorbox{black}{{\HUGE\textcolor{white}{\textbf{\MakeUppercase{Yi Tian (Leo)}}}}} % First name
	
	\colorbox{black}{{\HUGE\textcolor{white}{\textbf{\MakeUppercase{Liu}}}}} % Last name
	
	\vspace{6pt}
	
	{\huge} % Career or current job title
\end{minipage}
\begin{minipage}[t]{0.275\textwidth} % 27.5% of the page width for the first row of icons
	\vspace{-\baselineskip} % Required for vertically aligning minipages
	
	% The first parameter is the FontAwesome icon name, the second is the box size and the third is the text
	% Other icons can be found by referring to fontawesome.pdf (supplied with the template) and using the word after \fa in the command for the icon you want
	\icon{MapMarker}{12}{San Diego, CA, US}\\
	\icon{Phone}{12}{+1 619 627 1990}\\
	\icon{At}{12}{\href{mailto:leoyitianliu@gmail.com}{leoyitianliu@gmail.com}}\\	
\end{minipage}
\begin{minipage}[t]{0.275\textwidth} % 27.5% of the page width for the second row of icons
	\vspace{-\baselineskip} % Required for vertically aligning minipages
	
	% The first parameter is the FontAwesome icon name, the second is the box size and the third is the text
	% Other icons can be found by referring to fontawesome.pdf (supplied with the template) and using the word after \fa in the command for the icon you want
	\icon{Globe}{12}{\href{https://leoyitianliu.com}{leoyitianliu.com}}\\
	\icon{Github}{12}{\href{https://github.com/illidan333}{github.com/illidan333}}\\
	\icon{Twitter}{12}{\href{https://twitter.com/@leoyitianliu}{@leoyitianliu}}\\
\end{minipage}

\vspace{0.5cm}

%----------------------------------------------------------------------------------------
%	INTRODUCTION, SKILLS AND TECHNOLOGIES
%----------------------------------------------------------------------------------------

\cvsect{Summary}

\begin{minipage}[t]{0.4\textwidth} % 40% of the page width for the introduction text
	\vspace{-\baselineskip} % Required for vertically aligning minipages
	13 years of work experience in IT\newline\newline
    7 years experience as an DevOps/SRE\newline\newline
    Specialized in troubleshooting and automation\newline\newline
    Rich experience in high-availability systems\newline\newline
\end{minipage}
\hfill % Whitespace between
\begin{minipage}[t]{0.5\textwidth} % 50% of the page for the skills bar chart
	\vspace{-\baselineskip} % Required for vertically aligning minipages
	\begin{barchart}{5.5}
		\baritem{Infra as Code}{70}
		\baritem{Config as Code}{90}
		\baritem{Linux}{90}
        \baritem{Full Stack Development}{80}
	\end{barchart}
\end{minipage}

%----------------------------------------------------------------------------------------
%	EXPERIENCE
%----------------------------------------------------------------------------------------

\cvsect{Experience}

\begin{entrylist}
	\entry
		{2024 - Now}
		{Site Reliability Engineer}
		{SHEIN Group - SHEIN Technology LLC}
		{
            Responsible for the reliability and availability of the system \newline

            - Building scripts/tools to automate the operation process including disk analysis and changes, network analysis and update. \newline
            - Practicing chaos engineering and performed drills to ensure the fail over redundancy can work when the primary system is malfunctioning. \newline
            - Incident response and root cause analysis in production environment. \newline
            - Managing Kubernetes clusters deployments, scaling, version upgrades, resource demand analysis and adjustment. \newline
            - Managing the infrastructure for web service components including Kafka and RabbitMQ. \newline
        }
	\entry
		{2024}
		{DevOps Build Engineer}
		{Tencent - Timbre Games}
		{
            - Automated game build tasks on TeamCity, enhancing efficiency and reliability in build processes. \newline
            - Designed and implemented an auto-scaling build farm, integrating Perforce, TeamCity, and AWS (EC2, EBS) to dynamically scale build agents based on demand.\newline
            - Reduced build queue time by 80\%, significantly improving build turnaround time and developer productivity.\newline
            - Configured seamless integration among key components to optimize workflows and resource utilization.\newline

            \textit{Projects: Auto-scaling build farm\newline
            Configuring the integration among different compoments like Perforce, Teamcity and AWS (EC2 and EBS) to allow 
            auto scaling build agents on demand. Reduced build queue time by 80\%\newline}
        }
	\entry
		{2022 -- 2023}
		{Senior DevOps Engineer}
		{Offworld Industries}
		{
            - Delivered a web application called \textit{Robomerge} to boost development productivity by 90\%\newline
            - Improved website and game web service reliability by 40\% through enhanced development and operations.\newline
            - Saved hundreds of team hours monthly by integrating an open-source utility into Unreal Engine workflows for streamlined conflict resolution.\newline

            \textit{Projects: Robomerge\newline
            Integrated an open-source utility in the Unreal Engine to the team’s process to automatically merge code branches and warn people on Slack about any conflicts. It saved the game team hundreds of hours every month to deal with merge conflicts\newline}
        }
    \entry
		{2019 -- 2022}
		{DevOps Site Reliability Engineer}
		{Readymode}
		{
            - Automated infrastructure provisioning with Python, Ansible, and AutoIT, reducing setup time by 80\% and saving 100+ engineering hours weekly.\newline
            - Built a deployment tool to help the company to increase system reliability by 200\%\newline
            - Redesigned the Committee deployment tool, boosting system reliability by 200\% and improving visibility, flexibility, and rollback capabilities.\newline

            \textit{Project: Infrastructure Provisioning Automation\newline
            The sale team was doing too well, so our customers grew too rapidly. We needed more servers.  
            We are asked to manually install OS and software dependencies repeatedly to keep up with our growth. I fully committed to the task assigned, but I learn how to automate the process whenever I have time. After a while, I managed make the process 80\% faster by automating with Python, Ansible and AutoIT.\newline\newline}
            \textit{Project: Committee II\newline
            As the number of our servers grow rapidly, code deployment became very unreliable. Things break often. We need to efficiently and reliably deploy the code with the flexibility on who gets what version. I rewrote our code deployment tool called “Committee”. It was faster and kept records in the database. The new deployment tool was faster and more reliable with visibility, flexibility and easy to rollback.\newline}
        }
	\entry
		{2019}
		{Software Development Engineer}
		{Volkswagen - PayByPhone}
		{
            - Improved a .NET application a bit with test-drive development in C\#\newline
            - Expanded the functionality of an API service by Integrating AWS web services\newline 
            - Enhanced some AWS cloud infrastructure orchestration with terraform\newline
        }
	\entry
		{2016 -- 2019}
		{Intermediate Software Development Engineer}
		{XenCALL}
		{
            - Led a team to build a sales pipeline system to help users to grow their sales by 200\%\newline
            - Led a team to create a gamification system to train 100k+ users to learn software features\newline
            - Led a team to create an advance search system to improve lead info search speed by 70+ \%\newline
            - Built a licensing system to help the company to collect 30+ thousands dollars missing revenue\newline
            - Created a system to track tech support agents stat to reduce management workload by 80\%\newline

            \textit{Project: Gamification Training System\newline
            Led a team to create a gamified training system with quests for learning complex software features. 
            Collected user data for troubleshooting and assessment. 
            Accelerated user onboarding and improved training engagement. \newline\newline}
            \textit{Project: Sales Pipeline Optimization\newline
            Developed a fast-performing application to track sales leads and bottlenecks. 
            Used data caching, aggregation, and query optimization for speed. 
            Improved customer ability to manage sales pipelines effectively.\newline}
        }
	\entry
		{2014 -- 2016}
		{Junior Software Development Engineer}
		{XenCALL}
		{
            - Developed a payment website to help over 1000+ clients to make payment online easily\newline
            - Developed call center data reports to improved data visibility to gain 100+ user complements\newline
            \newline
            \textit{Project: Tech Support Performance Management System\newline
            Created a system to track tech support agents' work hours and calculate wages accurately. 
            Ensured fair pay, motivated punctuality, and clarified promotion criteria.
            \newline\newline}
            \textit{Project: Agent, Call Log, and Dialer Reports\newline
            Developed reports to extract and present call center data with filtering, sorting, and heatmap features. 
            Improved data visibility, enabling better decision-making.\newline}
        }
    \entry
		{2012 -- 2014}
		{Interactive Test Engineer}
		{Gaming Laboratories International}
		{
            - Performed source code review on gambling software to ensure fidelity\newline
            - Developed software programs to automate repetitive testing procedures\newline
            - Performed manual QA tests and regression tests to detect software bugs\newline
        }
\end{entrylist}

%----------------------------------------------------------------------------------------
%	EDUCATION
%----------------------------------------------------------------------------------------

\cvsect{Education}

\begin{entrylist}
	\entry
		{2008 -- 2012}
		{Bachelor's Degree of Computer Engineering (Software Option)}
		{University of British Columbia}
		{}
\end{entrylist}

\end{document}
