%%%%%%%%%%%%%%%%%%%%%%%%%%%%%%%%%%%%%%%%%
% Developer CV
% LaTeX Template
% Version 1.0 (28/1/19)
%
% This template originates from:
% http://www.LaTeXTemplates.com
%
% Authors:
% Jan Vorisek (jan@vorisek.me)
% Based on a template by Jan Küster (info@jankuester.com)
% Modified for LaTeX Templates by Vel (vel@LaTeXTemplates.com)
%
% License:
% The MIT License (see included LICENSE file)
%
%%%%%%%%%%%%%%%%%%%%%%%%%%%%%%%%%%%%%%%%%

%----------------------------------------------------------------------------------------
%	PACKAGES AND OTHER DOCUMENT CONFIGURATIONS
%----------------------------------------------------------------------------------------

\documentclass[9pt]{developercv} % Default font size, values from 8-12pt are recommended

%----------------------------------------------------------------------------------------

\begin{document}

%----------------------------------------------------------------------------------------
%	TITLE AND CONTACT INFORMATION
%----------------------------------------------------------------------------------------

\begin{minipage}[t]{0.45\textwidth} % 45% of the page width for name
	\vspace{-\baselineskip} % Required for vertically aligning minipages
	
	% If your name is very short, use just one of the lines below
	% If your name is very long, reduce the font size or make the minipage wider and reduce the others proportionately
	\colorbox{black}{{\HUGE\textcolor{white}{\textbf{\MakeUppercase{Yi Tian (Leo)}}}}} % First name
	
	\colorbox{black}{{\HUGE\textcolor{white}{\textbf{\MakeUppercase{Liu}}}}} % Last name
	
	\vspace{6pt}
	
	{\huge} % Career or current job title
\end{minipage}
\begin{minipage}[t]{0.275\textwidth} % 27.5% of the page width for the first row of icons
	\vspace{-\baselineskip} % Required for vertically aligning minipages
	
	% The first parameter is the FontAwesome icon name, the second is the box size and the third is the text
	% Other icons can be found by referring to fontawesome.pdf (supplied with the template) and using the word after \fa in the command for the icon you want
	\icon{MapMarker}{12}{California, United States}\\
	\icon{Phone}{12}{+1 619 627 1990}\\
	\icon{At}{12}{\href{mailto:leoyitianliu@gmail.com}{leoyitianliu@gmail.com}}\\	
\end{minipage}
\begin{minipage}[t]{0.275\textwidth} % 27.5% of the page width for the second row of icons
	\vspace{-\baselineskip} % Required for vertically aligning minipages
	
	% The first parameter is the FontAwesome icon name, the second is the box size and the third is the text
	% Other icons can be found by referring to fontawesome.pdf (supplied with the template) and using the word after \fa in the command for the icon you want
	\icon{Globe}{12}{\href{https://leoyitianliu.com}{leoyitianliu.com}}\\
	\icon{Github}{12}{\href{https://github.com/illidan333}{github.com/illidan333}}\\
	\icon{Twitter}{12}{\href{https://twitter.com/@leoyitianliu}{@leoyitianliu}}\\
\end{minipage}

\vspace{0.5cm}

%----------------------------------------------------------------------------------------
%	INTRODUCTION, SKILLS AND TECHNOLOGIES
%----------------------------------------------------------------------------------------

\cvsect{Summary}

\begin{minipage}[t]{0.4\textwidth} % 40% of the page width for the introduction text
	\vspace{-\baselineskip} % Required for vertically aligning minipages
    6 years of DevOps Engineer/SRE\newline\newline
	5 years of Full-stack developer\newline\newline
    Mastering both cloud and on-premise\newline\newline
    Experience in large distributed systems\newline\newline
\end{minipage}
\hfill % Whitespace between
\begin{minipage}[t]{0.5\textwidth} % 50% of the page for the skills bar chart
	\vspace{-\baselineskip} % Required for vertically aligning minipages
	\begin{barchart}{5.5}
		\baritem{High Reliability and Scalabilty}{70}
		\baritem{Infra / Config as Code}{90}
		\baritem{Linux}{90}
        \baritem{Full Stack Development}{80}
	\end{barchart}
\end{minipage}

%----------------------------------------------------------------------------------------
%	EXPERIENCE
%----------------------------------------------------------------------------------------

\cvsect{Experience}

\begin{entrylist}
	\entry
		{Nov2024-Now}
		{Site Reliability Engineer}
		{Shein, San Diego, California, United States}
		{
            Ensure the reliability and performance of SHEIN’s large-scale global infrastructure, supporting over 50,000+ servers, petabyte-level data systems, and millions of requests daily.\newline

			- Designed and implemented object-oriented applications to manage and optimize complex infrastructure workflows and rule sets.\newline
            - Developed automation tools for disk and network operations, improving operational efficiency and reducing manual intervention by over 70\% across thousands of nodes.\newline
            - Enhanced system reliability and availability, achieving 99.998\% uptime through proactive on-call response, system testing, chaos drills, and resilience engineering.\newline
            - Led incident management for high-severity production events, achieving SLA goals of MTTA < 1 min, MTTD < 5 min, and MTTR < 10 min, minimizing user impact and service disruption.\newline
            - Managed large-scale Kubernetes clusters with thousands of nodes, ensuring seamless deployments, auto-scaling, version upgrades, and resource allocation with zero downtime.\newline
			- Operated critical infrastructure components (MySQL, Redis, Kafka) under high throughput, optimizing performance and achieving 50\% reduction in average query latency.\newline
			- Implemented a comprehensive monitoring and alerting stack, enabling real-time observability and early detection, resulting in a 40\% reduction in false positives.\newline
			- Scaled infrastructure on Microsoft Azure across multiple global regions, achieving 30\% cost optimization while maintaining system performance and availability.\newline
			- Performed advanced system operations across Unix/Linux servers, using shell scripting and CLI tools to manage infrastructure at scale.\newline\newline
			\textit{Project: NSG Management App (OOD-based)\newline
			Designed and implemented an object-oriented application to manage Azure NSG (Network Security Group) rules across global infrastructure. The system scans existing NSG configurations, evaluates new rule requests, and suggests the most optimal changes with minimal redundancy. Significantly improved access control efficiency and reduced manual errors.\newline}\newline
            \textit{Project: Project: Automated Large-Scale Disk Replacement Application\newline
            Built a scalable automation framework to evaluate and reallocate disk resources across 50,000+ servers and saved the organization 80K USD per month.}\newline\newline
        }
	\entry
		{Dec2024--Aug2024}
		{Build Engineer}
		{Tencent - Timbre Games, Vancouver, British Columbia, Canada}
		{
            - Automated game build tasks on TeamCity, enhancing efficiency and reliability in build processes. \newline
            - Designed and implemented an auto-scaling build farm, integrating Perforce, TeamCity, and AWS (EC2, EBS) to dynamically scale build agents based on demand.\newline
            - Reduced build queue time by 80\%, significantly improving build turnaround time and developer productivity.\newline
            - Configured seamless integration among key components to optimize workflows and resource utilization.\newline

            \textit{Projects: Auto-scaling build farm\newline
            Configuring the integration among different compoments like Perforce, Teamcity and AWS (EC2 and EBS) to allow 
            auto scaling build agents on demand. Reduced build queue time by 80\%\newline}
        }
	\entry
		{Jan2022--Dec2023}
		{DevOps Engineer}
		{Offworld Industries, New Westminster, British Columbia, Canada}
		{
            - Delivered a web application called \textit{Robomerge} to boost development productivity by 90\%\newline
            - Improved website and game web service reliability by 40\% through enhanced development and operations.\newline
            - Saved hundreds of team hours monthly by integrating an open-source utility into Unreal Engine workflows for streamlined conflict resolution.\newline

            \textit{Projects: Robomerge\newline
            Integrated an open-source utility in the Unreal Engine to the team’s process to automatically merge code branches and warn people on Slack about any conflicts. It saved the game team hundreds of hours every month to deal with merge conflicts\newline}
        }
    \entry
		{Oct2019--Jan2022}
		{Site Reliability Developer}
		{Readymode, Vancouver, British Columbia, Canada}
		{
            - Automated infrastructure provisioning with Python, Ansible, and AutoIT, reducing setup time by 80\% and saving 100+ engineering hours weekly.\newline
            - Built a deployment tool to help the company to increase system reliability by 200\%\newline
            - Redesigned the Committee deployment tool, boosting system reliability by 200\% and improving visibility, flexibility, and rollback capabilities.\newline

            \textit{Project: Infrastructure Provisioning Automation\newline
            The sale team was doing too well, so our customers grew too rapidly. We needed more servers.  
            We are asked to manually install OS and software dependencies repeatedly to keep up with our growth. I fully committed to the task assigned, but I learn how to automate the process whenever I have time. After a while, I managed make the process 80\% faster by automating with Python, Ansible and AutoIT.\newline\newline}
            \textit{Project: Committee II\newline
            As the number of our servers grow rapidly, code deployment became very unreliable. Things break often. We need to efficiently and reliably deploy the code with the flexibility on who gets what version. I rewrote our code deployment tool called “Committee”. It was faster and kept records in the database. The new deployment tool was faster and more reliable with visibility, flexibility and easy to rollback.\newline}
        }
		\entry
		{Jul2019--Oct2019}
		{Software Development Engineer}
		{Volkswagen - PayByPhone, Vancouver, British Columbia, Canada}
		{
            - Improved a .NET application a bit with test-drive development in C\#\newline
            - Expanded the functionality of an API service by Integrating AWS web services\newline 
            - Enhanced some AWS cloud infrastructure orchestration with terraform\newline\newline
			\textit{Project: Parking Patroller software\newline
            Developped an application to help the Patroller to check if someone has paid for parking
            Providing parking info like grace period and time parked and time to expire
            Giving tickets to parking violators \newline}
        }
	\entry
		{Jan2016 -- Jul2019}
		{Intermediate Software Development Engineer}
		{XenCALL, Vancouver, British Columbia, Canada}
		{
            - Led a team to build a sales pipeline system to help users to grow their sales by 200\%\newline
            - Led a team to create a gamification system to train 100k+ users to learn software features\newline
            - Led a team to create an advance search system to improve lead info search speed by 70+ \%\newline
            - Built a licensing system to help the company to collect 30+ thousands dollars missing revenue\newline
            - Created a system to track tech support agents stat to reduce management workload by 80\%\newline

            \textit{Project: Gamification Training System\newline
            Led a team to create a gamified training system with quests for learning complex software features. 
            Collected user data for troubleshooting and assessment. 
            Accelerated user onboarding and improved training engagement. \newline\newline}
            \textit{Project: Sales Pipeline Optimization\newline
            Developed a fast-performing application to track sales leads and bottlenecks. 
            Used data caching, aggregation, and query optimization for speed. 
            Improved customer ability to manage sales pipelines effectively.\newline}
        }
	\entry
		{Mar2014--Jan2016}
		{Junior Software Development Engineer}
		{XenCALL, Vancouver, British Columbia, Canada}
		{
            - Developed a payment website to help over 1000+ clients to make payment online easily\newline
            - Developed call center data reports to improved data visibility to gain 100+ user complements\newline
            \newline
            \textit{Project: Tech Support Performance Management System\newline
            Created a system to track tech support agents' work hours and calculate wages accurately. 
            Ensured fair pay, motivated punctuality, and clarified promotion criteria.
            \newline\newline}
            \textit{Project: Agent, Call Log, and Dialer Reports\newline
            Developed reports to extract and present call center data with filtering, sorting, and heatmap features. 
            Improved data visibility, enabling better decision-making.\newline}
        }
    \entry
		{Jul2012--Jan2014}
		{Interactive Test Engineer}
		{Gaming Laboratories International, Burnaby, British Columbia, Canada}
		{
            - Performed source code review on gambling software to ensure fidelity\newline
            - Developed software programs to automate repetitive testing procedures\newline
            - Performed manual QA tests and regression tests to detect software bugs\newline\newline
            \textit{Project: C++ program to automate testing process\newline
            Developed a desktop tool to automate some game testing process. 
            Automatically run slot machine games and verify if the result is expected.\newline}
        }
\end{entrylist}

%----------------------------------------------------------------------------------------
%	EDUCATION
%----------------------------------------------------------------------------------------

\cvsect{Education}

\begin{entrylist}
	\entry
		{2008 -- 2012}
		{Bachelor's Degree of Computer Engineering (Software Option)}
		{University of British Columbia}
		{}
\end{entrylist}
\begin{entrylist}
	\entry
		{2016}
		{Unity VR Developer Workshop}
		{Circuit Steam}
		{}
\end{entrylist}
\begin{entrylist}
	\entry
		{2018}
		{Introduction to Agile and Scrum}
		{Agile 42}
		{}
\end{entrylist}
\begin{entrylist}
	\entry
		{2019}
		{Introduction on Deep Learning}
		{CloudXLab}
		{}
\end{entrylist}

\cvsect{Volunteer Experience}

\begin{entrylist}
	\entry
		{2024 -- Now}
		{Technical Manager}
		{Career Up Club}
		{Website development and maintenance, events preparing and hosting}
\end{entrylist}

\end{document}