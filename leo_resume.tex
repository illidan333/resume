%%%%%%%%%%%%%%%%%%%%%%%%%%%%%%%%%%%%%%%%%
% Developer CV
% LaTeX Template
% Version 1.0 (28/1/19)
%
% This template originates from:
% http://www.LaTeXTemplates.com
%
% Authors:
% Jan Vorisek (jan@vorisek.me)
% Based on a template by Jan Küster (info@jankuester.com)
% Modified for LaTeX Templates by Vel (vel@LaTeXTemplates.com)
%
% License:
% The MIT License (see included LICENSE file)
%
%%%%%%%%%%%%%%%%%%%%%%%%%%%%%%%%%%%%%%%%%

%----------------------------------------------------------------------------------------
%	PACKAGES AND OTHER DOCUMENT CONFIGURATIONS
%----------------------------------------------------------------------------------------

\documentclass[9pt]{developercv} % Default font size, values from 8-12pt are recommended

%----------------------------------------------------------------------------------------

\begin{document}

%----------------------------------------------------------------------------------------
%	TITLE AND CONTACT INFORMATION
%----------------------------------------------------------------------------------------

\begin{minipage}[t]{0.45\textwidth} % 45% of the page width for name
	\vspace{-\baselineskip} % Required for vertically aligning minipages
	
	% If your name is very short, use just one of the lines below
	% If your name is very long, reduce the font size or make the minipage wider and reduce the others proportionately
	\colorbox{black}{{\HUGE\textcolor{white}{\textbf{\MakeUppercase{Yi Tian (Leo)}}}}} % First name
	
	\colorbox{black}{{\HUGE\textcolor{white}{\textbf{\MakeUppercase{Liu}}}}} % Last name
	
	\vspace{6pt}
	
	{\huge} % Career or current job title
\end{minipage}
\begin{minipage}[t]{0.275\textwidth} % 27.5% of the page width for the first row of icons
	\vspace{-\baselineskip} % Required for vertically aligning minipages
	
	% The first parameter is the FontAwesome icon name, the second is the box size and the third is the text
	% Other icons can be found by referring to fontawesome.pdf (supplied with the template) and using the word after \fa in the command for the icon you want
	\icon{Linkedin}{12}{\href{https://www.linkedin.com/in/leoyitianliu/}{leoyitianliu}}\\
	\icon{Phone}{12}{+1 619 627 1990}\\
	\icon{At}{12}{\href{mailto:leoyitianliu@gmail.com}{leoyitianliu@gmail.com}}\\	
\end{minipage}
\begin{minipage}[t]{0.275\textwidth} % 27.5% of the page width for the second row of icons
	\vspace{-\baselineskip} % Required for vertically aligning minipages
	
	% The first parameter is the FontAwesome icon name, the second is the box size and the third is the text
	% Other icons can be found by referring to fontawesome.pdf (supplied with the template) and using the word after \fa in the command for the icon you want
	\icon{Globe}{12}{\href{https://leoyitianliu.com}{leoyitianliu.com}}\\
	\icon{Github}{12}{\href{https://github.com/illidan333}{github.com/illidan333}}\\
	\icon{Twitter}{12}{\href{https://twitter.com/@leoyitianliu}{@leoyitianliu}}\\
\end{minipage}

\vspace{0.5cm}

%----------------------------------------------------------------------------------------
%	INTRODUCTION, SKILLS AND TECHNOLOGIES
%----------------------------------------------------------------------------------------

\cvsect{Summary}

\begin{minipage}[t]{0.4\textwidth} % 40% of the page width for the introduction text
	\vspace{-\baselineskip} % Required for vertically aligning minipages
    Experienced Site Reliability Engineer and Full-Stack Developer with 10+ years in building and operating large-scale, distributed systems. Passionate about improving reliability, scalability, and developer velocity through automation, object-oriented design, and deep infrastructure knowledge—across both cloud and on-premise environments.
\newline\newline
\end{minipage}
\hfill % Whitespace between
\begin{minipage}[t]{0.5\textwidth} % 50% of the page for the skills bar chart
	\vspace{-\baselineskip} % Required for vertically aligning minipages
		\begin{barchart}{5.5}
		\baritem{Distributed Systems \& Scalability}{90}
		\baritem{Kubernetes \& Terraform}{90}
		\baritem{Python / Go / Java}{85}
		\baritem{CI/CD Automation}{85}
		\baritem{Incident Response \& Observability}{80}
\end{barchart}
\end{minipage}

%----------------------------------------------------------------------------------------
%	EXPERIENCE
%----------------------------------------------------------------------------------------

\cvsect{Experience}

\begin{entrylist}
	\entry
		{Nov2024-Now}
		{Site Reliability Engineer}
		{Shein, San Diego, California, United States}
		{
            Ensure the reliability and performance of SHEIN’s large-scale global infrastructure, supporting over 50,000+ servers, petabyte-level data systems, and millions of requests daily.\newline

			- Designed and implemented object-oriented applications to manage and optimize complex infrastructure workflows and rule sets.\newline
            - Developed automation tools for disk and network operations, improving operational efficiency and reducing manual intervention by over 70\% across thousands of nodes.\newline
            - Enhanced system reliability and availability, achieving 99.998\% uptime through proactive on-call response, system testing, chaos drills, and resilience engineering.\newline
            - Managed large-scale Kubernetes clusters with thousands of nodes, ensuring seamless deployments, auto-scaling, version upgrades, and resource allocation with zero downtime.\newline

			\textit{Project: NSG Management App (OOD-based)\newline
			Designed and implemented an object-oriented application to manage Azure NSG (Network Security Group) rules across global infrastructure. The system scans existing NSG configurations, evaluates new rule requests, and suggests the most optimal changes with minimal redundancy. Significantly improved access control efficiency and reduced manual errors.\newline}\newline
            \textit{Project: Project: Automated Large-Scale Disk Replacement Application\newline
            Built a scalable automation framework to evaluate and reallocate disk resources across 50,000+ servers and saved the organization 80K USD per month.}\newline\newline
        }
	\entry
		{Jan2024--Aug2024}
		{Build and Release Engineer}
		{Tencent - Timbre Games, Vancouver, British Columbia, Canada}
		{
            Committed to enhancing development productivity by improving the CI/CD build pipeline\newline

            \textit{Projects: Auto-scaling build farm\newline
            Configuring the integration among different compoments like Perforce, Teamcity and AWS (EC2 and EBS) to allow 
            auto scaling build agents on demand. Reduced build queue time by 80\%\newline}
        }
	\entry
		{Jan2022--Dec2023}
		{DevOps Engineer}
		{Offworld Industries, New Westminster, British Columbia, Canada}
		{
            Enhanced development productivity by improving the build pipeline, server infrastructure and DevOps processes\newline

            \textit{Projects: Robomerge\newline
            Integrated an open-source utility in the Unreal Engine to the team’s process to automatically merge code branches and warn people on Slack about any conflicts. It saved the game team hundreds of hours every month to deal with merge conflicts\newline}
        }
    \entry
		{Oct2019--Jan2022}
		{Site Reliability Developer}
		{Readymode(XenCALL), Vancouver, British Columbia, Canada}
		{
			Scaled a VOIP web application from 500 users to 30000 users by resolving challenging scalability issues\newline

            - Automated infrastructure provisioning with Python, Ansible, and AutoIT, reducing setup time by 80\% and saving 100+ engineering hours weekly.\newline
            - Built a deployment tool to help the company to increase system reliability by 200\%\newline
            - Redesigned the Committee deployment tool, boosting system reliability by 200\% and improving visibility, flexibility, and rollback capabilities.\newline

            \textit{Project: Infrastructure Provisioning Automation\newline
            The sale team was doing too well, so our customers grew too rapidly. We needed more servers.  
            We are asked to manually install OS and software dependencies repeatedly to keep up with our growth. I fully committed to the task assigned, but I learn how to automate the process whenever I have time. After a while, I managed make the process 80\% faster by automating with Python, Ansible and AutoIT.\newline\newline}
            \textit{Project: Committee II\newline
            As the number of our servers grow rapidly, code deployment became very unreliable. Things break often. We need to efficiently and reliably deploy the code with the flexibility on who gets what version. I rewrote our code deployment tool called “Committee”. It was faster and kept records in the database. The new deployment tool was faster and more reliable with visibility, flexibility and easy to rollback.\newline}
        }
		\entry
		{Jul2019--Oct2019}
		{Software Development Engineer}
		{Volkswagen - PayByPhone, Vancouver, British Columbia, Canada}
		{
			\textit{
            Developped an application to help the Patroller to check if someone has paid for parking
            Providing parking info like grace period and time parked and time to expire
            Giving tickets to parking violators \newline}
        }
	\entry
		{Jan2016 -- Jul2019}
		{Intermediate Software Development Engineer}
		{XenCALL(Readymode), Vancouver, British Columbia, Canada}
		{
            - Led a team to build a sales pipeline system to help users to grow their sales by 200\%\newline
            - Led a team to create a gamification system to train 100k+ users to learn software features\newline
            - Led a team to create an advance search system to improve lead info search speed by 70+ \%\newline
            - Built a licensing system to help the company to collect 30+ thousands dollars missing revenue\newline
            - Created a system to track tech support agents stat to reduce management workload by 80\%\newline

            \textit{Project: Gamification Training System\newline
            Led a team to create a gamified training system with quests for learning complex software features. 
            Collected user data for troubleshooting and assessment. 
            Accelerated user onboarding and improved training engagement. \newline\newline}
            \textit{Project: Sales Pipeline Optimization\newline
            Developed a fast-performing application to track sales leads and bottlenecks. 
            Used data caching, aggregation, and query optimization for speed. 
            Improved customer ability to manage sales pipelines effectively.\newline}
        }
	\entry
		{Mar2014--Jan2016}
		{Junior Software Development Engineer}
		{XenCALL(Readymode), Vancouver, British Columbia, Canada}
		{
            - Developed a secure payment portal used by 10k+ customers, reducing support tickets by 40\%\newline
            - Developed call center data reports to improved data visibility to gain 100+ user complements\newline
            \newline
            \textit{Project: Tech Support Performance Management System\newline
            Created a system to track tech support agents' work hours and calculate wages accurately. 
            Ensured fair pay, motivated punctuality, and clarified promotion criteria.
            \newline\newline}
            \textit{Project: Agent, Call Log, and Dialer Reports\newline
            Developed reports to extract and present call center data with filtering, sorting, and heatmap features. 
            Improved data visibility, enabling better decision-making.\newline}
        }
    \entry
		{Jul2012--Jan2014}
		{Interactive Test Engineer}
		{Gaming Laboratories International, Burnaby, British Columbia, Canada}
		{Performed and automated gambling game tests to catch bugs and ensure regulation compliance\newline}
\end{entrylist}

%----------------------------------------------------------------------------------------
%	EDUCATION
%----------------------------------------------------------------------------------------

\cvsect{Education}

\begin{entrylist}
	\entry
		{2008 -- 2012}
		{Bachelor's Degree of Computer Engineering (Software Option)}
		{University of British Columbia}
		{}
\end{entrylist}

\end{document}